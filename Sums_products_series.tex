\documentclass[../Maxima_Workbook.tex]{subfiles}

\begin{document}
	
\chapter{Sums, products and series}

\section{Sums and products}

\subsection{Sums}

\subsubsection{Introduction}

A consecutive sum, with the index running over a range of consecutive integers, can be created with function \emph{sum}. I can be displayed in sigma notation, simplified and evaluated. Sums can also be differentiated or integrated, and they can be subject to limits.

\lz A selective sum, with the index only taking selected indices from a list, is created with function \emph{lsum}.


\subsubsection{Sum: consecutive indices} \mbox{}

\lzz \hyt{sum}{\tcr{\emph{sum (expr, $ i, i_0, i_1 $)}}} \hfill \tcr{[function]}\index{sum}

\lz Builds a "continuous" summation of \emph{expr} (evaluated) as the summation index i (not evaluated) runs from $i_0$ to $i_1$ (both evaluated). Both a noun form and a sum that by simplification and evaluation cannot be resolved are displayed in sigma notation. By setting $ i_1 $ to \emph{inf} for infinity we obtain a series, see section \ref{SP1}.

\lz \begin{small}
\color{blue} \leqn
\begin{lstlisting}
<@\tcr{(\%i1)}@   'sum(1/k!,k,0,4);
\end{lstlisting}
\vspace{-5mm} \[\tag*{\tcr{\ttfamily (\%o1)}} \sum_{k=1}^{4} \frac{1}{k!} \]
\vspace{-6mm} \begin{lstlisting}
<@\tcr{(\%i2)}@   sum(1/k!,k,0,4);
\end{lstlisting}
\vspace{-5mm} \[\tag*{\tcr{\ttfamily (\%o2)}} \frac{65}{24} \]
\vspace{-7mm} \begin{lstlisting}
<@\tcr{(\%i3)}@   sum(1/k!,k,1,n);
\end{lstlisting}
\vspace{-4mm} \[\tag*{\tcr{\ttfamily (\%o3)}} \sum_{k=1}^{n} \frac{1}{k!} \]

\begin{lstlisting}
<@\tcr{(\%i4)}@   sum (a[i], i, 1, 5);
\end{lstlisting}
\vspace{-6mm} \[\tag*{\tcr{\ttfamily (\%o4)}} a_1+a_2+a_3+a_4+a_5 \]
\vspace{-10mm} \begin{lstlisting}
<@\tcr{(\%i5)}@   sum (a(i), i, 1, 5);
<@\tcr{(\%o5)}@ 		   a(5) + a(4) + a(3) + a(2) + a(1)
\end{lstlisting}
\color{black} \reqn
\end{small}


\paragraph{Simplification}

\subparagraph{Simpsum} \mbox{}

\lz Some basic rules are applied automatically to simplify sums. More rules are activated by setting the flag \emph{simpsum} to \emph{true}.

\lzz \hyt{simpsum}{\tcr{\emph{simpsum}}} \qquad \tcr{default: \emph{false}} \hfill \tcr{[option variable]}\index{simpsum}

\lz When \emph{simpsum} is set, the result of a sum is simplified. This simplification may sometimes be able to produce a closed form. See section \ref{SP1} for the application to series.

\lz \begin{small}
\color{blue} \leqn
\begin{lstlisting}
<@\tcr{(\%i1)}@   sum (2^k + k^2, k, 0, n);
\end{lstlisting}
\vspace{-5mm} \[\tag*{\tcr{\ttfamily (\%o1)}} \sum_{k=0}^{n}{\left(2^{k}+k^2\right)} \]
\vspace{-5mm} \begin{lstlisting}
<@\tcr{(\%i2)}@   sum (2^k + k^2, k, 0, n), simpsum;
\end{lstlisting}
\vspace{-4mm} \[\tag*{\tcr{\ttfamily (\%o2)}} 2^{n+1}+{{2\,n^3+3\,n^2+n}\over{6}}-1 \]
\color{black} \reqn
\end{small} \vspace{-5mm} 

\subparagraph{Simplify\_sum} \mbox{}

\lz Package \emph{simplify\_sum} contains function \emph{simplify\_sum} which is more powerful in finding closed forms than setting flag \emph{simpsum}.

\lzz \hyt{simplify\_sum}{\tcr{\emph{simplify\_sum (expr)}}} \hfill \tcr{[function of \emph{simplify\_sum}]}\index{simplify\_sum}

\lz Tries to simplify all sums appearing in \emph{expr} to a closed form.

\lz \begin{small}
\color{blue} \leqn
\begin{lstlisting}
<@\tcr{(\%i1)}@   load(simplify_sum);
\end{lstlisting}
\vspace{-3mm} \begin{lstlisting}[basicstyle=\footnotesize]
<@\tcr{( \%o1)}@    C:/maxima-5.40.0/../share/maxima/5.40.0/share/solve_rec/simplify_sum.mac
\end{lstlisting}
\vspace{-3mm} \begin{lstlisting}
<@\tcr{(\%i2)}@   sum(2^k+k^2,k,0,n);
\end{lstlisting}
\vspace{-4mm} \[\tag*{\tcr{\ttfamily (\%o2)}} \sum_{k=0}^{n}{\left. {{2}^{k}}+{{k}^{2}}\right.} \]
\vspace{-5mm} \begin{lstlisting}
<@\tcr{(\%i3)}@   simplify_sum(%);
\end{lstlisting}
\vspace{-4mm} \[\tag*{\tcr{\ttfamily (\%o3)}} 2^{n+1}+{{2\,n^3+3\,n^2+n}\over{6}}-1 \]
\color{black} \reqn
\end{small} \vspace{-5mm}

\subsubsection{Lsum: selected indices}

\lz \hyt{lsum}{\tcr{\emph{lsum (expr, i, L)}}} \hfill \tcr{[function]}\index{sum}

\lz Represents the sum of \emph{expr} for each index contained in the list L. A noun form is returned, if the L does not evaluate to a list. All arguments except for i are evaluated.

\lz \begin{small}
\color{blue} \leqn
\begin{lstlisting}
<@\tcr{(\%i1)}@   'lsum (x^i, i, [1, 2, 7]);
\end{lstlisting}
\vspace{-5mm} \[\tag*{\tcr{\ttfamily (\%o1)}} \sum_{i\operatorname{in}[1,2,7]}{\left. {{x}^{i}}\right.} \]
\vspace{-7mm} \begin{lstlisting}
<@\tcr{(\%i2)}@   lsum (x^i, i, [1, 2, 7]);
\end{lstlisting}
\vspace{-5mm} \[\tag*{\tcr{\ttfamily (\%o2)}} {{x}^{7}}+{{x}^{2}}+x \]
\color{black} \reqn
\end{small} \vspace{-6mm}

\subsubsection{Nusum}

\lz \hyt{nusum}{\tcr{\emph{nusum (expr, $ i , i_0, i_1 $)}}} \hfill \tcr{[function]}\index{nusum}

\lz This is a \emph{new} sum function, more powerful than \hyl{sum}{\emph{sum}}, capable of simplification and of finding more closed forms, of series as well.

\lz The noun form of \emph{nusum} is not displayed in sigma notation. However, sigma notation is used for the return value, when \emph{nusum} cannot simplify \emph{expr}. 

\lz Let's construct an example, where \emph{sum} throws the towel. (For the last computation, which redoes the summation, see \hyl{unsum}{\emph{unsum}}).

\lz \begin{small}
\color{blue} \leqn
\begin{lstlisting}
<@\tcr{(\%i1)}@   sum (i^4*4^i/binomial(2*i,i), i, 0, n), simpsum;
\end{lstlisting}
\vspace{-4mm} \[\tag*{\tcr{\ttfamily (\%o1)}} \sum_{i=0}^{n}{\left. \frac{{{i}^{4}}\, {{4}^{i}}}{\begin{pmatrix}2 i\\
		i\end{pmatrix}}\right.} \]
\vspace{-5mm} \begin{lstlisting}
<@\tcr{(\%i2)}@   nusum (i^4*4^i/binomial(2*i,i), i, 0, n);
\end{lstlisting}
\vspace{-4mm} \[\tag*{\tcr{\ttfamily (\%o2)}} \frac{2 \left( n+1\right) \, \left( 63 {{n}^{4}}+112 {{n}^{3}}+18 {{n}^{2}}-22 n+3\right) \, {{4}^{n}}}{693 \begin{pmatrix}2 n\\
	n\end{pmatrix}}-\frac{2}{231} \]
\vspace{-5mm} \begin{lstlisting}
<@\tcr{(\%i3)}@   unsum(%,n);
\end{lstlisting}
\vspace{-4mm} \[\tag*{\tcr{\ttfamily (\%o3)}} \frac{{{n}^{4}}\, {{4}^{n}}}{\begin{pmatrix}2 n\\
	n\end{pmatrix}} \]
\color{black} \reqn
\end{small} \vspace{-4mm}

\subsubsection{Differentiating and integrating sums}

Sums can be differentiated and integrated.

\lz \begin{small}
\color{blue} \leqn
\begin{lstlisting}
<@\tcr{(\%i1)}@   s:sum((x-x0)^k,k,1,n);
\end{lstlisting}
\vspace{-4mm} \[\tag*{\tcr{\ttfamily (\%o1)}} \sum_{k=1}^{n} (x-x_0)^k \]
\vspace{-6mm} \begin{lstlisting}
<@\tcr{(\%i2)}@   'diff(s,x) = diff(s,x);
\end{lstlisting}
\vspace{-4mm} \[\tag*{\tcr{\ttfamily (\%o2)}} \frac{d}{dx} \sum_{k=1}^{n} (x-x_0)^k = \sum_{k=1}^{n} k \, (x-x_0)^{k-1} \]
\vspace{-4mm} \begin{lstlisting}
<@\tcr{(\%i3)}@   'integrate(s,x) = integrate(s,x);
\end{lstlisting}
\vspace{-4mm} \[\tag*{\tcr{\ttfamily (\%o3)}} \int \sum_{k=1}^{n} (x-x_0)^k \; dx = \sum_{k=1}^{n} \frac{(x-x_0)^{k+1}}{k+1} \]
\color{black} \reqn
\end{small} \vspace{-3mm}

\subsubsection{Limits of sums}

\lzz Sums can be subject to limits.

\subsubsection{Unsum: undoing a sum}

\lz \hyt{unsum}{\tcr{\emph{unsum (expr, n)}}} \hfill \tcr{[function]}\index{unsum}

\lz This function is kind of magic. It undoes a definite (i.e. finite) summation done with \hyl{sum}{\emph{sum}} or \hyl{nusum}{\emph{nusum}} and having a symbol as the (finite) upper bound. The undo is done by taking the closed form and returning the expression under the sigma sign. The argument \emph{expr} must be the closed form of a definite summation and n the symbol designating its (finite) upper bound. The lower bound of the original sum does not matter, it can be an integer or a symbol. (This implies, that unsum cannot find it out, either.) See \emph{nusum} for a more sophisticated example, if you still don't believe it.

\lz \begin{small}
\color{blue} \leqn
\begin{lstlisting}
<@\tcr{(\%i1)}@   nusum(i^2,i,0,n);
\end{lstlisting}
\vspace{-4mm} \[\tag*{\tcr{\ttfamily (\%o1)}} \frac{n\, \left( n+1\right) \, \left( 2 n+1\right) }{6} \]
\vspace{-5mm} \begin{lstlisting}
<@\tcr{(\%i2)}@   unsum(%,n);
\end{lstlisting}
\vspace{-4mm} \[\tag*{\tcr{\ttfamily (\%o2)}} {{n}^{2}} \]
\vspace{-5mm} \begin{lstlisting}
<@\tcr{(\%i3)}@   nusum(i^2,i,m,n);
\end{lstlisting}
\vspace{-4mm} \[\tag*{\tcr{\ttfamily (\%o3)}} \frac{\left( n-m+1\right) \, \left( 2 {{n}^{2}}+2 m n+n+2 {{m}^{2}}-m\right) }{6} \]
\vspace{-5mm} \begin{lstlisting}
<@\tcr{(\%i2)}@   unsum(%,n);
\end{lstlisting}
\vspace{-4mm} \[\tag*{\tcr{\ttfamily (\%o2)}} {{n}^{2}} \]
\color{black} \reqn
\end{small} \vspace{-7mm}

\subsection{Products}

\section{Series}

\subsection{Introduction}

Maxima contains functions \emph{powerseries} and \emph{taylor} for finding the series of differentiable functions. It also has tools such as \emph{nusum} capable of finding the closed form of some series. Operations such as addition and multiplication work as usual on series. This section presents the global variables which control the expansion.

\lz Series, including power series and truncated taylor expansions, can be differentiated and integrated.

\subsection{Sum or nusum with infinite upper bound}\label{SP1}

In Maxima a series is constructed using functions \hyl{sum}{\emph{sum}} or \hyl{nusum}{\emph{nusum}} with the upper bound set to \emph{inf} for infinity. In case of \emph{sum}, all simplification procedures described for finite sums can be used for series as well. Function \emph{nusum} is more powerful concerning simplification, and it achieves closed forms more often than \emph{sum}. We give \emph{sum} a try with a simple geometric series.

\lz \begin{small}
\color{blue} \leqn
\begin{lstlisting}
<@\tcr{(\%i1)}@   'sum(x^i,i,0,inf);
\end{lstlisting}
\vspace{-5mm} \[\tag*{\tcr{\ttfamily (\%o1)}} \en \sum_{i=0}^{\infty }{\left. {{x}^{i}}\right.} \]
\vspace{-5mm} \begin{lstlisting}
<@\tcr{(\%i2)}@   simpsum:true$  sum(x^i,i,0,inf);
	<@\tcr{Is |x|-1 positive, negative or zero?}@ pos;
	<@\tcr{sum: sum is divergent.  --- an error.}@	
<@\tcr{(\%i3)}@   sum(x^i,i,0,inf);
	<@\tcr{Is |x|-1 positive, negative or zero?}@ neg;
\end{lstlisting}
\vspace{-4mm} \[\tag*{\tcr{\ttfamily (\%o1)}} \frac{1}{1-x} \]
\color{black} \reqn
\end{small} \vspace{-4mm}

Function \emph{nusum} has a different strategy of informing the user about the conditions for convergence and divergence of the series.

\lz \begin{small}
\color{blue} \leqn
\begin{lstlisting}
<@\tcr{(\%i1)}@   nusum(x^i,i,0,inf)$
<@\tcr{(\%i2)}@   ChangeSign(ChangeSign(%,2),2,2);
\end{lstlisting}
\vspace{-5mm} \[\tag*{\tcr{\ttfamily (\%o2)}} \frac{{{x}^{\infty +1}}}{x-1}+\frac{1}{1-x} \]
\color{black} \reqn
\end{small} \vspace{-4mm}

\subsection{Power series}

\lz \hyt{powerseries}{\tcr{\emph{powerseries (expr, x, a)}}} \hfill \tcr{[function]}\index{powerseries}

\lz Returns the general form of the power series expansion for \emph{expr} in the variable x about the point a (which may be \emph{inf}). Each time Maxima returns a power series expansion, it creates a new summation index, starting with $ i1, i2, \dots $ 

\lz If \emph{powerseries} is unable to expand \emph{expr}, \emph{taylor} may be used to give the first several terms of the series.

\lz \begin{small}
\color{blue} \leqn
\begin{lstlisting}
<@\tcr{(\%i1)}@   powerseries(sin(x),x,0);
\end{lstlisting}
\vspace{-4mm} \[\tag*{\tcr{\ttfamily (\%o1)}} \sum_{\mathit{i1}=0}^{\infty }{\left. \frac{{{\left( -1\right) }^{\mathit{i1}}}\, {{x}^{2 \, \mathit{i1}+1}}}{\left( 2 \, \mathit{i1}+1\right) !}\right.} \]
\color{black} \reqn
\end{small} \vspace{-2mm} 

\lz When \emph{verbose} is \emph{true}, \emph{powerseries} prints progress messages before returning the result.

\lz \begin{small}
\color{blue} \leqn
\begin{lstlisting}
<@\tcr{(\%i2)}@   verbose:true$  powerseries(log(sin(x)/x),x,0);
trigreduce: failed to expand.
\end{lstlisting}
\vspace{-2mm} \[ \log{\left( \frac{\sin{(x)}}{x}\right) }\]
\vspace{-4mm} \begin{lstlisting}
trigreduce: try again after applying rule:
\end{lstlisting}
\[ \log{\left( \frac{\sin{(x)}}{x}\right) }=\int {\left. \frac{\frac{d}{d x} \frac{\sin{(x)}}{x}}{\frac{\sin{(x)}}{x}}dx\right.} \]
\vspace{-4mm} \begin{lstlisting}
powerseries: first simplification returned 
\end{lstlisting}
\[ -\int_{0}^{x}{\left. \frac{\csc{\left( \mathit{g494}\right) }\, \sin{\left( \mathit{g494}\right) }-\mathit{g494}\, \cos{\left( \mathit{g494}\right) }\, \csc{\left( \mathit{g494}\right) }}{\mathit{g494}}d\mathit{g494}\right.} \]
\vspace{-4mm} \begin{lstlisting}
powerseries: first simplification returned  
\end{lstlisting}
\[ -\frac{\mathit{g494}\, \cot{\left( \mathit{g494}\right) }-1}{\mathit{g494}} \]
\vspace{-4mm} \begin{lstlisting}
powerseries: attempt rational function expansion of  
\end{lstlisting}
\[ \frac{1}{\mathit{g494}}\]
\vspace{-4mm} \[\tag*{\tcr{\ttfamily (\%o3)}} \sum_{\mathit{i2}=1}^{\infty }{\left. \frac{{{\left( -1\right) }^{\mathit{i2}}}\, {{2}^{2 \, \mathit{i2}-1}}\, \operatorname{bern}\left( 2 \, \mathit{i2}\right) \, {{x}^{2 \, \mathit{i2}}}}{\mathit{i2}\, \left( 2 \, \mathit{i2}\right) !}\right.} \]
\color{black} \reqn
\end{small} \vspace{-2mm}

\lz The advanced running index of the g-variable generated by Maxima indicates that during computation the preceding ones have already been used internally.

\subsection{Taylor and Laurent series expansion}

\lz \hyt{taylor}{\tcr{\emph{taylor (expr, $  x, a, p \; \glangle, \; 'asymp \; \grangle $) \gbar}}} \hfill \tcr{[function]}\index{taylor}

\lz \tcr{\emph{taylor (expr, $ [x_1, \dots,x_n], \gpal a \gbar [a_1,\dots,a_n] \gpar, \gpal p \gbar [p_1,\dots,p_n] \gpar $) \gbar}}

\tcr{\emph{taylor (expr, $ [[x_1, \dots,x_n], \gpal a \gbar [a_1,\dots,a_n] \gpar, p \; \glangle, \; 'asymp\grangle] $) \gbar}}

\tcr{\emph{taylor (expr, $ [[x_1, \dots,x_n], a, \gpal p \gbar [p_1,\dots,p_n] \gpar \; \glangle, \; 'asymp \grangle] $) \gbar}}

\lz \tcr{\emph{taylor (expr, $ [x_1, a_1, p_1], \dots, [x_n, a_n, p_n] $) \gbar}}

\tcr{\emph{taylor (expr, $ x_1, a_1, p_1, \dots, x_n, a_n, p_n $)}}

\lz This is the general form of function \emph{taylor}. We will explain the single-variable and the multi-variable forms separately and then the \emph{'asymp} option.

\subsubsection{Single-variable form}

\lz \hyt{taylor}{\tcr{\emph{taylor (expr, x, a, p)}}} \hfill \tcr{[function]}\index{taylor}

\lz This basic form of \emph{taylor} expands the expression \emph{expr} in a truncated Taylor or Laurent series in the variable x around the point a, containing terms through $ (x - a)^p $. Maxima precedes the output of a Taylor expansion by a tag /T/ directly after the output tag. (In wxMaxima this is not done, if \emph{taylor} appears on the right side of an assignment.) This indicates that Maxima uses a special internal representation for this type of data. (The CRE form is yet another special internal data format, tagged with /R/ in Maxima output.)

\lz \begin{small}
\color{blue} \leqn
\begin{lstlisting}
<@\tcr{(\%i1)}@   taylor(sqrt(x+1),x,0,3);
\end{lstlisting}
\vspace{-6mm} \[ \tag*{\tcr{\ttfamily (\%o3) /T/}} 1+\frac{x}{2}-\frac{{{x}^{2}}}{8}+\frac{{{x}^{3}}}{16}+\mbox{...} \]
\color{black} \reqn
\end{small} \vspace{-4mm} 

\lz We can evaluate both the original function and Taylor expansions of various orders at a point near a with function \hyl{at}{\emph{at}} to see how the approximation proceeds.

\lz \begin{small}
\color{blue} \leqn
\begin{lstlisting}
<@\tcr{(\%i1)}@   t1:taylor(sqrt(x+1),x,0,1);
	t2:taylor(sqrt(x+1),x,0,2); 
	t3:taylor(sqrt(x+1),x,0,3); 
	t5:taylor(sqrt(x+1),x,0,5);
	at([t1,t2,t3,t5,sqrt(x+1)],x=0.3);
\end{lstlisting}
\vspace{-4mm} \[ \tag*{\tcr{\ttfamily (\%o1) /T/}} 1+\frac{x}{2}+\mbox{...} \]
\vspace{-6mm} \[ \tag*{\tcr{\ttfamily (\%o2) /T/}} 1+\frac{x}{2}-\frac{{{x}^{2}}}{8}+\mbox{...} \]
\vspace{-6mm} \[ \tag*{\tcr{\ttfamily (\%o3) /T/}} 1+\frac{x}{2}-\frac{{{x}^{2}}}{8}+\frac{{{x}^{3}}}{16}+\mbox{...} \]
\vspace{-6mm} \[ \tag*{\tcr{\ttfamily (\%o4) /T/}} 1+\frac{x}{2}-\frac{{{x}^{2}}}{8}+\frac{{{x}^{3}}}{16}-\frac{5 {{x}^{4}}}{128}+\frac{7 {{x}^{5}}}{256}+\mbox{...} \]
\vspace{-5mm} \begin{lstlisting}
<@\tcr{(\%o5)}@    [1.15, 1.13875, 1.1404375, 1.1401875390625, 1.140175425099138]
\end{lstlisting}
\color{black} \reqn
\end{small} \vspace{-2mm} 

\subsubsection{Multi-variable form}

\lz \tcr{\emph{taylor (expr, $ [x_1, \dots,x_n], \gpal a \gbar [a_1,\dots,a_n] \gpar, \gpal p \gbar [p_1,\dots,p_n] \gpar $)}}

\lz This basic multi-variable form of \emph{taylor} expands \emph{expr} in the variables $ x_1, \dots,x_n $ about the point ($ a_1,\dots,a_n $), up to combined powers of p or up to combined powers of $ \max (p_i) $ for $ i=1,\dots,n $. (Note that here p is not equal to the number of terms as it is in the single-variable form.) If $ a $ is identical for all $ i $, it can be given as a single simple variable instead of a list. Thus, $ a $ means the point $ \underbrace{(a, \dots, a)}_{n \; times} $.

\lzz \tcr{\emph{taylor (expr, $ [x_1, a_1, p_1], \dots, [x_n, a_n, p_n] $)}}

\lz (The square brackets can be omitted.) This form is not only syntactically different from the preceding one, but it also gives a different result, because \emph{expr} is expanded up to the power $ p_i $ for variable $ x_i, \; i=1,\dots,n $. Furthermore, terms are not factored according to combined powers as in the preceding form, but according to powers of the first, second, third, $ \dots $ variable.

\lz \begin{small}
\color{blue} \leqn
\begin{lstlisting}
<@\tcr{(\%i1)}@   taylor(sin(x+y),[x,y],0,5);
	expand(%);
\end{lstlisting}
\vspace{-4mm} \[ \tag*{\tcr{\ttfamily (\%o1) /T/}} y+x-\frac{{{x}^{3}}+3 y\, {{x}^{2}}+3 {{y}^{2}} x+{{y}^{3}}}{6}+\mbox{...} \]
\vspace{-5mm} \[ \tag*{\tcr{\ttfamily (\%o2)}} -\frac{{{y}^{3}}}{6}-\frac{x\, {{y}^{2}}}{2}-\frac{{{x}^{2}} y}{2}+y-\frac{{{x}^{3}}}{6}+x \]
\vspace{-2mm} \begin{lstlisting}
<@\tcr{(\%o3)}@   taylor(sin(x+y),[x,0,2],[y,0,3]);
	expand(%);
\end{lstlisting}
\vspace{-4mm} \[ \tag*{\tcr{\ttfamily (\%o3) /T/}} y-\frac{{{y}^{3}}}{6}+\mbox{...}+\left( 1-\frac{{{y}^{2}}}{2}+\mbox{...}\right)  x+\left( -\frac{y}{2}+\frac{{{y}^{3}}}{12}+\mbox{...}\right) \, {{x}^{2}}+\mbox{...} \]
\vspace{-5mm} \[ \tag*{\tcr{\ttfamily (\%o4) }} \frac{{{x}^{2}}\, {{y}^{3}}}{12}-\frac{{{y}^{3}}}{6}-\frac{x\, {{y}^{2}}}{2}-\frac{{{x}^{2}} y}{2}+y+x \]
\color{black} \reqn
\end{small} \vspace{-5mm} 

\subsubsection{Option 'asymp}

The option \emph{'asymp} can be applied to both the single- and the multi-variable form of \emph{taylor}. It returns an expansion of \emph{expr} in negative powers of $ x - a $. The highest order term is $ (x - a)^{-n} $.

\subsubsection{Option variables}

\lz \tcr{\emph{taylordepth}} \qquad \tcr{default value: 3} \hfill \tcr{[option variable]}\index{taylordepth}

\lz If in \emph{taylor (expr, x, a, p)} the expression \emph{expr} is of the form f(x)/g(x) and g(x) has no terms up to degree p, \emph{taylor} attempts to expand g(x) up to degree $ 2 \, p $. If there are still no non-zero terms, \emph{taylor} doubles the degree of the expansion of g(x) so long as the degree of the expansion is less than or equal to $ 2^{taylordepth} p $.

\end{document}
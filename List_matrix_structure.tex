\documentclass[../Maxima_Workbook.tex]{subfiles}

\begin{document}
	
\chapter{List, matrix, structure}

\section{List}\label{DS1}

\subsection{makelist}

\lz \hyt{makelist}{\tcr{\emph{makelist ($ \glangle expr \glangle, n \grangle \grangle \, $) $ \; \gbar $}}} \hfill \tcr{[function]}\index{makelist} \\
\hyt{makelist}{\tcr{\emph{makelist ($ expr, i \, \glangle, i_0 \grangle i_{max} \glangle, step \grangle \, $) $ \; \gbar $}}} \\
\hyt{makelist}{\tcr{\emph{makelist ($ \gpal expr \, \gbar \, [expr_1,\dots,expr_n] \gpar, x, list \, $)}}}

\lz This is a very powerful function to create lists from expressions and/or other lists. There are three general forms, each of them with some possible variations.

\lz The first form ...

\lz The second form ...

\lz For an example see the example to the discrete plot of \emph{plot2d}.

\lz The third form returns a list whose elements are evaluations of \emph{expr} or sublists being evaluations of $ expr_1,\dots,expr_n $. These expressions are functions of variable x, which takes its values running through \emph{list}. \emph{makelist}'s return value has as many elements as \emph{list} has, i.e. \emph{length(list)} elements. For \emph{j} running from 1 through \emph{length(list)}, the $ j^{th} $ element of the list returned is given by \emph{ev(expr,x=list[j])} or \emph{ev($ [expr_1,\dots,expr_n] $,x=list[j])}. 

\lz For an example see the second example to the function \emph{rk} implementing the Runge-Kutta method for numerically solving a first order ODE.

\subsection{create\_list}

\section{Matrix}


\section{Structure}


\end{document}
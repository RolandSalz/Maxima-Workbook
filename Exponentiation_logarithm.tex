\documentclass[../Maxima_Workbook.tex]{subfiles}

\begin{document}

\chapter{Roots, exponential and logarithmic functions}

\section{Roots}

\lz \tcr{\emph{sqrt(expr)}} \hfill \tcr{[function]}\index{sqrt}

\lz Returns the square root of \emph{expr}.

\subsection{Internal representation}

Square roots, as well as n-th roots in general, are represented internally as expressions with the exponentiation operator to a rational exponent. So \emph{sqrt(x)} is represented by $ x^{(1/2)} $.


\subsection{Simplification}

\hyt{radexpand}{\tcr{\emph{radexpand}}} \qquad \tcr{default: \emph{true}} \hfill \tcr{[option variable]}\index{radexpand}

\lz \hyt{domain}{\tcr{\emph{domain}}} \qquad \tcr{default: \emph{real}} \hfill \tcr{[option variable]}\index{domain}

\lz When \emph{radexpand} is set to its default value \emph{true} and \emph{domain} to its default value \emph{real}, sqrt(x\^\ 2) is simplified to abs(x). 

\lz When \emph{radexpand} is \emph{all} or assume(x>0) has been executed, nth roots of factors which are powers of n are pulled outside of the root. E.g. $ sqrt (16*x^2) $ is simplified to $ 4x $, and $ sqrt(x^2) $ to x.

\lz When \emph{radexpand} is \emph{false} or (\emph{radexpand} is \emph{true} and \emph{domain} is set to \emph{complex}), $ sqrt(x^2) $ will not be simplified.

\lzz \hyt{rootscontract}{\tcr{\emph{rootscontract (expr)}}} \hfill \tcr{[function]}\index{rootscontract}

\lz \hyt{rootsconmode}{\tcr{\emph{rootsconmode}}} \qquad \tcr{default: \emph{true}} \hfill \tcr{[option variable]}\index{rootsconmode}

\lz \emph{rootscontract} converts products (or quotients) of roots into roots of products (or quotients). For example, \emph{rootscontract $ (sqrt(x)*y^{(3/2)}) $} yields \emph{sqrt($ x \, y^3 $)}.

\lz \emph{rootsconmode} controls, how \emph{rootscontract} is applied.

\subsection{Roots of negative real or of complex numbers}

Maxima allows negative real numbers, and more generally, complex numbers as arguments of any n-th root. If a real root exists, it is returned, otherwise Maxima computes the principal complex root. This, however, in certain cases will not be accomplished by a single command. Maxima does not automatically simplify complex numbers, so it may be that the expression is simplified only partially and will be returned containing a noun form. Further simplification can be achieved with \hyl{rectform}{rectform}.

\lz \begin{small}
\color{blue} \leqn
\begin{lstlisting}
<@\tcr{(\%i1)}@   (-8)^(1/3)
<@\tcr{(\%o1)}@			        -2
<@\tcr{(\%i2)}@   sqrt(-4)
<@\tcr{(\%o2)}@			        2i
<@\tcr{(\%i3)}@   (-8)^(1/4);
\end{lstlisting}
\vspace{-7mm} \[\tag*{\tcr{\ttfamily (\%o3)}} {{\left( -1\right) }^{\frac{1}{4}}}\, {{8}^{\frac{1}{4}}} \]
\vspace{-10mm} \begin{lstlisting}
<@\tcr{(\%i4)}@   float(%);
\end{lstlisting}
\vspace{-7mm} \[\tag*{\tcr{\ttfamily (\%o4)}} 1.681792830507429 \; {{\left( -1\right) }^{\frac{1}{4}}} \]
\vspace{-10mm} \begin{lstlisting}
<@\tcr{(\%i5)}@   (-1)^(1/4)
\end{lstlisting}
\vspace{-7mm} \[\tag*{\tcr{\ttfamily (\%o5)}} {{\left( -1\right) }^{\frac{1}{4}}} \]
\vspace{-10mm} \begin{lstlisting}
<@\tcr{(\%i6)}@   rectform(%);
\end{lstlisting}
\vspace{-6mm} \[\tag*{\tcr{\ttfamily (\%o6)}} \frac{\% i}{\sqrt{2}}+\frac{1}{\sqrt{2}} \]
\vspace{-8mm} \begin{lstlisting}
<@\tcr{(\%i7)}@   float(%);
\end{lstlisting}
\vspace{-5mm} \[\tag*{\tcr{\ttfamily (\%o7)}} 0.7071067811865475i+0.7071067811865475 \]
\color{black} \reqn
\end{small} \vspace{-7mm}

\subsubsection{Computing all n complex roots}

In the preceeding section we saw that Maxima computes only the principal root with the exponentiation operator to a rational exponent. If we want to compute all n (pairwise different) complex roots, we can use function \hyl{solve}{solve}. The principal root will usually be the last one in the list returned. As an example, we want to compute all three cubic roots of $ 110+74i $.

\lz \begin{small}
\color{blue} \leqn
\begin{lstlisting}
<@\tcr{(\%i1)}@   float(rectform(solve(z^3=110+74*%i,z)));
<@\tcr{(\%o1)}@     [z=3.830127018922193i-3.366025403784439, z=-4.830127018922193i-1.633974596215562, z=0.9999999999999999i+5.000000000000001]
<@\tcr{(\%i2)}@   expand((5+%i)^3);
<@\tcr{(\%o2)}@			     74i+110
\end{lstlisting}
\color{black} \reqn
\end{small}

\lz Comparing this with

\lz \begin{small}
\color{blue}
\begin{lstlisting}
<@\tcr{(\%i1)}@   float(rectform((110+74*%i)^(1/3)));
<@\tcr{(\%o2)}@		     0.9999999999999997i+5.0
\end{lstlisting}
\color{black}
\end{small}

\lz we notice that the expressions for the principal root differ slightly, apparantly due to the internal use of different algorithms.

\section{Exponential function}

\lz \hyt{exp}{\tcr{\emph{rexp (expr)}}} \hfill \tcr{[function]}\index{exp}

\lz \emph{exp} ist die natürliche Exponentialfunktion. Maxima vereinfacht \verb|exp(x)|sofort zu $ e^x $.

\subsection{Simplification}

\lz \hyt{radcan}{\tcr{\emph{radcan (expr)}}} \hfill \tcr{[function]}\index{radcan}

\lz Die Funktion \emph{radcan} vereinfacht Ausdrücke, die die Exponentialfunktion, den Logarithmus und Wurzeln enthalten.

\lz \begin{small}
	\color{blue} \leqn
\lz \begin{lstlisting}
<@\tcr{(\%i2)}@   (%e^x-1)/(1+%e^(x/2));
	radcan(%);
\end{lstlisting}
\vspace{-4mm} \[\tag*{\tcr{\ttfamily (\%o1)}} \frac{e^x-1}{e^{\frac{x}{2}+1}} \]
\vspace{-4mm} \[\tag*{\tcr{\ttfamily (\%o2)}} e^{\frac{x}{2}-1} \]
\color{black} \reqn
\end{small}

\lz \hyt{logsimp}{\tcr{\emph{logsimp}}} \qquad \tcr{default: \emph{true}} \hfill \tcr{[option variable]}\index{logsimp}

\lz Ist die Optionsvariable \emph{logsimp} gesetzt, wird eine Exponentialform \verb|%e^(r*log(x))| $ \equiv \enskip e^{r \, ln(x)} $ zu \verb|x^r| vereinfacht, falls $ r \in \Z $.

\lzz \hyt{\%e\_to\_numlog}{\tcr{\emph{e\_to\_numlog}}} \qquad \tcr{default: \emph{false}} \hfill \tcr{[option variable]}\index{\%e\_to\_numlog}

\lz Ist die Optionsvariable \emph{\%e\_to\_numlog} gesetzt, wird eine Exponentialform der Art \verb|%e^(r*log(x))| $ \equiv \enskip e^{r \, ln(x)} $ zu \verb|x^r| vereinfacht, falls $ r \in \Q $.

\lzz \hyt{demoivre}{\tcr{\emph{demoivre}}} \qquad \tcr{default: \emph{false}} \hfill \tcr{[option variable]}\index{demoivre}

\lz Ist die Optionsvariable \emph{demoivre} gesetzt, wird eine Exponentialform \verb|%e^(a+%i*b)| $ \equiv \enskip e^{a+i \, b} $ mit $ a, b \in \R $, also mit komplexem Exponenten in Standardform, mit der Euler'schen Formel zu \verb|%e^a*(cos(b)+%i*sin(b))| $ \equiv \enskip e^a \, (\cos b + i \, \sin b) $, also zu einem äquivalenten Ausdruck mit Kreisfunktionen, umgeformt.

\lz Die Optionsvariable \emph{exponentialize} führt die gegenteilige Umformung durch. Es können also nicht beide Optionsvariablen gleichzeitig gesetzt sein. Beide Umformungen können auch durch Funktionen gleichen Namens bewirkt werden, ohne daß die Optionsvariablen gesetzt sind.

\lz \begin{small}
\color{blue} \leqn
\begin{lstlisting}
<@\tcr{(\%i4)}@   %e^(a+ %i*b);
	%e^(a+ %i*b), demoivre:true;
	%, exponentialize:true;
	radcan(%);
\end{lstlisting}
\vspace{-4mm} \[\tag*{\tcr{\ttfamily (\%o1)}} e^{a+i \, b} \]
\vspace{-8mm} \[\tag*{\tcr{\ttfamily (\%o2)}} e^a \, (\cos b + i \, \sin b) \]
\vspace{-6mm} \[\tag*{\tcr{\ttfamily (\%o3)}} e^a \left( \frac{e^{i \, b} - e^{-i \, b}}{2} + \frac{e^{i \, b} + e^{-i \, b}}{2} \right) \]
\vspace{-4mm} \[\tag*{\tcr{\ttfamily (\%o4)}} e^{a+i \, b} \]
\color{black} \reqn
\end{small}

\lzz \hyt{\%emode}{\tcr{\emph{\%emode}}} \qquad \tcr{default: \emph{true}} \hfill \tcr{[option variable]}\index{\%emode}

\lz Ist die Optionsvariable \emph{\%emode} gesetzt, wird eine Exponentialform \verb|%e^(%i*%pi*x)| $ \equiv \enskip e^{i \, \pi \, x} $ vereinfacht 

\lz - falls x eine ganze Zahl, ein ganzzahliges Vielfaches von 1/2, 1/3, 1/4 oder 1/6 oder eine Gleitkommazahl ist, die einer ganzen oder halbganzzahligen Zahl entspricht: nach der Euler'schen Formel zu einer komplexen Zahl in der Standardform \verb|cos(%pi*x)+%i*sin(%pi*x)| und dann wenn möglich weiter vereinfacht,

\lz - für andere rationale x zu einer Exponentialform \verb|%e^(%i*%pi*y)|, mit $ y = x-2k $ für ein $ k \in \N $, sodaß $ |y| < 1 $ ist.

\lz Eine Exponentialform \verb|%e^(%i*%pi*(x+y))| $ \equiv \enskip e^{i \, \pi \, (x+y)} $ wird zu $ e^{i \, \pi \, x} e^{i \, \pi \, y} $ umgeformt und dann der erste Faktor entsprechend vereinfacht, wenn y ein Polynom oder etwa eine trigonometrische Funktion ist, nicht jedoch, wenn y eine rationale Funktion ist.

\lz Wenn mit komplexen Zahlen in Polarkoordinatenform gerechnet werden soll, kann es hilfreich sein, \emph{\%emode} auf den Wert \emph{false} zu setzen.

\lzz \hyt{\%enumer}{\tcr{\emph{\%enumer}}} \qquad \tcr{default: \emph{false}} \hfill \tcr{[option variable]}\index{\%enumer}

\lz In an exponential form with floating point exponent, \%e is always evaluated to floating point, and therefore the whole form. If both \emph{\%enumer} and \emph{numer} are true, \%e is evaluated to floating point in any expression.

\end{document}